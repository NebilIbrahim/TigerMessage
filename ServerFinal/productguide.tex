\documentclass[12pt]{article}
\usepackage{setspace}
\usepackage{dirtree}
\doublespacing

\begin{document}
\section*{For Future Developers}

\subsection*{File Architecture and Important Files}

The main files appear as follows:

\dirtree{%
  .1 Server.
  .2 index.py.
  .2 dbopsAttempt.py.
  .2 mlops.py.
  .2 requirements.txt.
  .2 nltk.txt.
  .2 Procfile.
  .2 templates.
  .3 index.html.
  .3 chats.html.
  .2 static.
  .3 Image Files.
}

\subsubsection*{index.py}
index.py is the main server code. The bulk of it dedicated to the
background of the server setup and the socketIO event-handling code.
While many things in each event can be changed while keeping the
code modular, it is important that the socketIO interactions are
either unchanged, or changed in both front/backend.

\subsubsection*{index.html}
index.html is the main chat page. It sends, receives, and renders
messages, as well as prepping links and images with the appropriate
HTML tags. This is the other end of most of the SocketIO emits/ons
in \texttt{index.py}.

\subsubsection*{dbopsAttempt.py}
This is where all of the database code is handled. After it is
initialized with a database engine by flask, it handles all the
direct communication with the database.

\subsubsection*{mlops.py}
This file contains all of the code for the subject analysis
component of Tiger Message. It contains code to pull out objects and
subjects in sentences in addition to code which takes lists of these
and calculates a ``closeness'' between 0 and 1 using WordNet.

\subsubsection*{chats.html}
This contains the ``about'' page and the list of a user's chats.
There is one socketio event parsed by this site, which simply
refreshes the chat list.

\subsubsection*{Other Files}
The other files highlighted above are dependency information and run
commands for Heroku to correctly parse and run the server.

\subsection*{Key SocketIO Workflows}

There are several socketIO events passed back and forth between the
server and the client 

\subsubsection*{User Connects}
When a user connects, they go through the following process,
exchanging user information back and forth with the server.
\begin{center}
  \begin{tabular}{lcr}
    Client & & Server\\
    \hline
    connect & $\rightarrow$ & \\
           & $\leftarrow$ & token\\
    confirm & $\rightarrow$ \\
           &$\leftarrow$& not member\\
           & or &\\
           &$\leftarrow$& chat message.
  \end{tabular}
\end{center}

\subsubsection*{User Sends a Message}
If client1 sends a message in a chat with client2, the message is
processed by the server, and then distributed to all users in the
selected chat.
\begin{center}
  \begin{tabular}{lcccr}
    Client1 && Server && Client2\\
    \hline
    chat message &$\rightarrow$&&&\\
    &$\leftarrow$& chat message & $\rightarrow$&
  \end{tabular}
\end{center}

\subsubsection*{Changing a Chat's Name}
If client1 changes the name of a shared chat with client2, the
change is processed by the server and then confirmation to re-check
the chat's name is sent to each website.
\begin{center}
  \begin{tabular}{lcccr}
    Client1 && Server && Client2\\
    \hline
    change cname & $\rightarrow$& &&\\
    & $\leftarrow$ & confirm & $\rightarrow$&
  \end{tabular}
\end{center}

\subsubsection*{Sigle Event Interactions}
For certain things like restricting the messages to a subject,
re-displaying all chat messages, reloading pages, there is just a
single event that the server emits to all the necessary pages.

\end{document}
